\documentclass{ximera}  
\title{Introduction}  
\begin{document}  
\begin{abstract}  
We give a brief introduction to the course.
\end{abstract}  
\maketitle

Welcome to the MAT1630 Ximera pages. This is a work in progress that will be updated as the semester goes on. The goal of these documents is to introduce course concepts along with problems that you can practice online in a single location. 
This introduction is designed to give a brief overview of the course.

What this course is about:

\begin{enumerate}
	\item Solving mathematical problems with the help of a computer.
	\item Learning to tackle large problems by decomposing them into smaller sub-problems.
	\item Learning to recognize patterns and similarities between different types of problems.
	\item Practicing the process of abstracting and distilling a problem away from its original setting.
	\item Designing algorithms to solve problems.
\end{enumerate}

In this course we will be learning to use Python to help us solve problems. Disclaimer: this is not a programming course in the traditional sense. It is not expected that you be a Python expert at the end of the course. The goal is to learn enough Python to help you solve otherwise difficult problems. We will be skipping over several technical details in favor of practicality. 

This course is intended for students that have not yet taken a programming course, but will in the future and is not a substitute for a traditional C++, Java, Python, etc. programming course. 

The philosophy behind the creation of this course can be explained via the following example. A car is a useful tool. The typical individual will learn to drive and make good use of a car without having to know the exact details of how every part of it works. For most individuals, this will be enough. Those in need of more from car will go on to learn about its internal functions to understand and make use of its full capabilities. 

In this course, the computer is our tool. Our goal is to learn to wield it well enough to make good use of its basic capabilities using our mathematical knowledge and to motivate the programming courses that follow. The expertise and technical knowledge necessary to fully exploit the computer as a tool is reserved for a future course. 

\end{document}
