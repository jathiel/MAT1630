\documentclass{ximera}  
\title{Recursion}  
\begin{document}  
\begin{abstract}  
We introduce recursion and recursively defined functions.
\end{abstract}  
\maketitle

\section{Recursion}

Recursion is involved when an object's definition refers to itself. We illustrate this idea via an example.

The factorial function is defined over the nonnegative integers in the following way: if $n=0$, $n!=0$, otherwise $n!=n(n-1)(n-2)\cdots 1$. Note that in order to compute $n!$ for $n>2$, one first needs to compute $(n-1)!$. So we can define the factorial function one number at a time. We define $$n!=\begin{cases} 1 & \text{ if $n=0,1$}\\ n\cdot(n-1)! & \text{otherwise.}\end{cases}$$



\begin{verbatim}
==============================


==============================
\end{verbatim}


\begin{sageCell}
x
\end{sageCell}

\section{Problems}

Note that for the questions below, the hint contains the solution.

\section{Workspace}

\begin{sageCell}
# Use this cell to solve the above questions.
\end{sageCell}

\end{document}
