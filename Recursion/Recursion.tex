\documentclass{ximera}  
\title{Recursion}  
\begin{document}  
\begin{abstract}  
We introduce recursion and recursively defined functions.
\end{abstract}  
\maketitle

\section{Recursion}

Recursion is involved when an object's definition refers to itself. We illustrate this idea via an example.

The factorial function is defined over the nonnegative integers in the following way: $0!=0$, for $n\geq 1$, $n!=n(n-1)(n-2)\cdots 1$. Note that in order to compute $n!$ for $n>2$, one first needs to compute $(n-1)!$. So we can define the factorial function one number at a time. We let $$n!=\begin{cases} 1 & \text{if $n=0,1$}\\ n\cdot(n-1)! & \text{otherwise.}\end{cases}$$

This new way of defining $n!$ highlights the idea that to solve the problem of computing $n!$, one first must solve the similar, but smaller problem, of computing $(n-1)!$. Recursion typically comes up when this is the case.

Another example of recursion is given by the \link[Fibonacci sequence]{https://plus.maths.org/content/fibonacci-sequence-brief-introduction}. We define $f_n$, the $n$th Fibonacci number of $n\geq 0$ as $$f_n=\begin{cases} 1 & \text{if $n=0,1$}\\ f_{n-1}+f_{n-2} & \text{otherwise.}$$ Computing recursively, we see that this definition yields the sequence $1,1,2,3,5,8,\cdots$. That is, for every $n\geq 2$, $f_n$ is the sum of the previous two terms in the sequence.

\begin{sageCell}
def fib(n):
        if n < 2:
                return 1
        else:
                return fib(n-1) + fib(n-2)

fib(5)
\end{sageCell}

Every recursively defined function has two key ingredients:

	\begin{enuemrate}
	\item base case(s): these are known values of the function that will be used as a starting point for computing other values,
	\item inductive case: this is the rule for computing the function at unknown values using previously known values.
	\end{enumerate}

If either ingredient above is missing, then the function will not be defined correctly. (Can you think of why?)

Some advantages of using recursively defined functions:

	\begin{itemize}
		\item Recursively definied functions can be easy to implement in Python.
		\item In many cases recursion is the {\emph only} known way to define certain functions.

	\end{itemize}

A major dissadvantage of using a recursive definition is that it can require a lot of time to compute something. For example, computing $f_{100}$ in Python will take an extremely long period of time.

\begin{sageCell}
x
\end{sageCell}

\section{Problems}

Note that for the questions below, the hint contains the solution.

\section{Workspace}

\begin{sageCell}
# Use this cell to solve the above questions.
\end{sageCell}

\end{document}
