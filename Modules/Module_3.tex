\documentclass{ximera}
\title{List Comprehensions and Sets}
\begin{document}
\begin{abstract}
The goal is to understand how use list comprehensions and perform set.
\end{abstract}
\maketitle

\section{List Comprehensions}

A list comprehension allows you to create lists quickly using syntax that looks like set-builder notation.
\begin{verbatim}
============================== 
a = [k*k for k in range(5)]
b = [k for k in range(13) if k%2 == 0]
==============================
\end{verbatim}
        
The list \verb|a| contains the squares of 0 to 4 and \verb|b| contains all even numbers from 0 to 12.

\section{Sets and Set Operations}

Sets are similar to lists, but they cannot have repeated entries and have special set operations.

\begin{verbatim}
==============================
A = set([1,2,3,4,5])
B = set([3,4,5,6,7])
==============================
\end{verbatim}
\begin{enumerate}
        \item We can determine the membership of an element \verb|x| in a set \verb|A|:
\begin{verbatim}
==============================	
x in A
==============================
\end{verbatim}
        \item In the case of a finite set, we can compute its cardinality:
\begin{verbatim}
==============================
len(A)
==============================
\end{verbatim}
        \item We can determine if a set \verb|A| is a subset of another set \verb|B|:
\begin{verbatim}
==============================
A.issubset(B)
==============================
\end{verbatim}
        \item Unions:
\begin{verbatim}
==============================
A.union(B)
==============================
\end{verbatim}
        \item Intersection:
\begin{verbatim}
==============================
A.intersection(B)
==============================
\end{verbatim}
        \item Set Differences:
\begin{verbatim}
==============================
A.difference(B)
==============================
\end{verbatim}
    \end{enumerate}

\section{Problems}

    \begin{enumerate}
    	\item Use list comprehensions to create the following lists:
    	\begin{enumerate}
        	\item The list of all of the odd numbers from 1 to 100.
\begin{sageCell}
# Use this cell to solve the problem.
\end{sageCell}
        	\item The list of all letters in your name that are not `e'. (Hint: a string can be iterated over.)
\begin{sageCell}
# Use this cell to solve the problem.
\end{sageCell}
    	\end{enumerate}
    	\item Suppose $A=\{4,7,8,3,2\}$ and $B=\{8,7,5s,4\}$
        \begin{enumerate}
        	\item Compute $A\cap B$.
\begin{sageCell}
# Use this cell to solve the problem.
\end{sageCell}
        	\item Compute $A\cup B$.
\begin{sageCell}
# Use this cell to solve the problem.
\end{sageCell}
         	\item Compute $A-B$.
\begin{sageCell}
# Use this cell to solve the problem.
\end{sageCell}
         	\item Compute $|A\cup B|$.
\begin{sageCell}
# Use this cell to solve the problem.
\end{sageCell}
        \end{enumerate}
    \end{enumerate}
\end{document}
