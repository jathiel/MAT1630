\documentclass{ximera}  
\title{An introduction to Python and SageCell}  
\begin{document}
The goal is to learn t use some basic Python commands.
\begin{abstract}  
\end{abstract}  
\maketitle

\section{Basics}

\begin{enumerate}
\item Go to \url{https://trinket.io/} and create an account. This will help you create and manage your own work. Create a new trinket titled ``2440 Module 0" to begin today's activities. Type all of today's work into this trinket.
\item Basics:
    \begin{enumerate}
        \item The print command prints variables and strings to the screen.
        \begin{verbatim}
        print "Hello world!"
        x = 3
        print x
        print 3+4
        x = 5
        print x
        \end{verbatim}
        \item Comments let you insert text into your code that is not executed. This will help to remind you (and others) what your code does.
        \begin{verbatim}
        print #"Hi there!"
        \end{verbatim}
        \item Lists help you organize and manipulate data.
        \begin{verbatim}
        a = [1,2,3,5]
        print a
        print a[0]   #this picks out elements (starting with element 0)
        print len(a) #len is a function that gives the length of the list
        \end{verbatim}
        \item Mathematical functions like $e^x$, $\sin(x)$ have to be imported from another file. Use the following link to see what other functions are available: \url{https://docs.python.org/2/library/math.html}.
        \begin{verbatim}
        import math

        print math.exp(1)   #math.exp(x) is the exponential function
        print math.pi       #the constant 3.1415926...
        print math.cos(math.pi/4)
        \end{verbatim}

    \end{enumerate}
\item Write code to do the following:
    \begin{enumerate}
    \item Print a string that contains your full name.
    \item Print the length of the string from the previous question. (Hint: Strings are lists where each element is a single character.)
    \item Compute and print $3+4/2$. Are you surprised by the result?
    \item Compute and print $e^{-\pi}$.
    \item Are the trigonometric functions assuming that the input is in degrees or radians?
    \item Compute and print $2^{2/3}$. (Hint: You will need to find an appropriate function using the math library link listed above.)
    \end{enumerate}
\end{enumerate}


\section{Workspace}

\begin{sageCell}
# Use this cell to solve the above questions.
\end{sageCell}

\end{document}
