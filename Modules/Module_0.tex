\documentclass{ximera}  
\title{An introduction to Python and SageCell}  
\begin{document}
\begin{abstract}  
The goal is to learn to use some basic Python functions and SageCell.
\end{abstract}  
\maketitle

\section{Basics}

In this section we cover some basic Python syntax.

    \begin{enumerate}
        \item The print command prints variables and strings to the screen.
\begin{sageCell}
print("Hello world!")
x = 3
print(x)
print(3+4)
x = 5
print(x)
\end{sageCell}

	\item Comments let you insert text into your code that is not executed. This will help to remind you (and others) what your code does.
\begin{sageCell}
#"Hi there!"
x = 3
\end{sageCell}
	\item Lists help you organize and manipulate data.
\begin{sageCell}
a = [1,2,3,5]
print(a)
print(a[0])   #this picks out elements (starting with element 0)
print(len(a)) #len is a function that gives the length of the list
\end{sageCell}
		\item Mathematical functions like $e^x$, $\sin(x)$ have to be imported from another file. Use this \link[link]{https://docs.python.org/2/library/math.html} for more information.
\begin{sageCell}
import math

print(math.exp(1))   #math.exp(x) is the exponential function
print(math.pi)       #the constant 3.1415926...
print(math.cos(math.pi/4))
\end{sageCell}
    \end{enumerate}

\section{SageCell}

A SageCell is a web-interface where you can type in Python code and evaluate it without installing any special software or navigating to another page. SageCell runs a special version of Python 3 with some minor differences to the standard kernel. This should not present any serious difficulties in this course. Each problem in the next section has a SageCell after it. Be sure to type in your answer to each problem in the appropriate SageCell.

Below is an example of a SageCell. Click on the `Evaluate' button to see the result.

\begin{sageCell}
x = 3
print(2*x + 5)
\end{sageCell}

\section{Problems}
\begin{enumerate}
    \item Print a string that contains your full name.
\begin{sageCell}
# Use this cell to solve the problem.
\end{sageCell}

    \item Print the length of the string from the previous question. (Hint: Strings are lists where each element is a single character.)
\begin{sageCell}
# Use this cell to solve the problem.
\end{sageCell}

    \item Compute and print $3+4/2$. Are you surprised by the result?
\begin{sageCell}
# Use this cell to solve the problem.
\end{sageCell}

    \item Compute and print $e^{-\pi}$.
\begin{sageCell}
# Use this cell to solve the problem.
\end{sageCell}

    \item Are the trigonometric functions assuming that the input is in degrees or radians?
\begin{sageCell}
# Use this cell to solve the problem.
\end{sageCell}

    \item Compute and print $2^{2/3}$. (Hint: You will need to find an appropriate function using the math library link above.)
\begin{sageCell}
# Use this cell to solve the problem.
\end{sageCell}

\end{enumerate}
\end{document}
