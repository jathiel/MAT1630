\documentclass{ximera}  
\title{Python as a Calculator}  
\begin{document}  
\begin{abstract}  
Using Python for basic computations.
\end{abstract}  
\maketitle

\section{Expressions}

We can use Python as a fancy calculator to help us with basic computations. For example, \verb|2 + 3| will evaluate to \verb|5| as expected. The SageCell below is a place where you can practice evaluating lines of Python code. Use this cell to evaluate several mathematical expressions using the basic operators mentioned before.


\section{Comparisons}

Comparisons return either a \verb|True| or \verb|False| value depeding on whether or not the given statement holds. For example, \verb|3 < 1| would evaluate to \verb|False|. Comparisons will allow us to write code that reacts to different inputs. Use the cell below to evaluate several comparisons. Note that \verb|==| is how you check whether or not two quantities are equal (\verb|=| is reserved for assignment).

\section{Data Types}

We will be working with different data types, like integers, strings, and lists. Each one comes equipped with functions and operators whose output depends on the data types that they act on. We will make use of the following data types:

\begin{itemize}
	\item int - integers like 0, 3, or -5.
	\item float - numbers with fractional parts like 2.5
	\item bool - boolean values (\verb|True| or \verb|False|)
	\item char - a single character enclosed in single or double quotes, like `c' or `3' (note that the integer 3 is different from the character `3')
	\item string - a collection of characters, like `hello'
	\item lists and arrays (to be covered later)
\end{itemize}

\section{A SageCell}

Write your Python code in the cell below and click on the 'Evaluate' button below to see the result.

\begin{sageCell}

\end{sageCell}

\end{document}
