\documentclass{ximera}  
\title{Interest Problems}  
\begin{document}
\begin{abstract}  
We introduce some compound interest problems as an application.
\end{abstract}  
\maketitle

\section{Compound Interest}

Any financial agreement involving the borrowing of money typically includes an interest calculation. Interest can be thought of as the cost of borrowing from a lender and is usually computed as a percentage of the principal sum over a fixed period of time.

Compound interest involves the addition of the previously computed interest to the principal sum. That is, you compute the interest on the interest on the loan.

We use the following basic formula for compounding interest:

$$P' = P\cdot\left(1+\frac{r}{n}\right) + c $$ where
\begin{itemize}
	\item $P = $ principal sum,
	\item $r = $ annual interest rate,
	\item $n = $ number of compoundings per year,
	\item $c = $ contribution to the principal per compounding period, and 
	\item $P' = $ new principal sum after one compounding period.
\end{itemize}

Let $P_t$ be the value of the loan after $t$ compounding periods. When $c=0$, we can compute $P_t$ as $P\cdot\left(1+\frac{r}{n}\right)^{t}$. For the more general case that includes when $c\neq 0$, the following formula for a geometric sums will be helpful:

$$\sum_{k=0}^Nx^k = \frac{1-x^{N+1}}{1-x}.$$

Working backwards from the definition of $P_t$, we get that

\begin{align*}
	P_t & = P_{t-1}\cdot\left(1+\frac{r}{n}\right) + c\\
	    & = \left(P_{t-2}\cdot\left(1+\frac{r}{n}\right) + c\right)\cdot\left(1+\frac{r}{n}\right) + c\\
	    & = P_{t-2}\cdot\left(1+\frac{r}{n}\right)^2 + c\cdot\left(1+\frac{r}{n}\right) + c\\
	    & \cdots & \\
	    & = P\cdot\left(1+\frac{r}{n}\right)^t + c\sum_{k=0}^{t-1}\left(1+\frac{r}{n}\right)^k\\
	    & = P\cdot\left(1+\frac{r}{n}\right)^t + c\cdot\frac{1-\left(1+\frac{r}{n}\right)^t}{1-\left(1+\frac{r}{n}\right)}\\
	    & = P\cdot\left(1+\frac{r}{n}\right)^t - \frac{cn}{r}\cdot\left(1-\left(1+\frac{r}{n}\right)^t\right).
\end{align*}

So, 

\begin{fact}
  $$P_t = P\cdot\left(1+\frac{r}{n}\right)^t - \frac{cn}{r}\cdot\left(1-\left(1+\frac{r}{n}\right)^t\right).$$
\end{fact}

This formula for $P_t$ is useful for a variety of loan problems as will be seen below. It can be used to determine how much is owed after making payments of a fixed size and it can be used to compute the payment amount needed to pay off a loan over a fixed time period.

\section{Problems}

\begin{question}
Suppose you open a retirement account with \$1,000 that offers an annual interest rate of 6\% compounded monthly. (Round all answers to the nearest cent.)

How much is the account worth after 20 years with no monthly contributions? $\answer{3310.20}$

How much is the same account worth after 20 years if you make monthly contributions of \$50? $\answer{26412.25}$
\end{question}

\begin{question}
Which account is worth more after a period of 15 years?
\begin{itemize}
	\item A: A savings account earning an annual interest rate of 2.5\% compounded monthly having an initial investment of \$1,000 with no monthly contributions.
	\item B: A savings account earning an annual interst rate of 2.5\% compounded monthly having an initial investment of \$0 with \$50 in monthly contributions.
\end{itemize}
	\begin{multipleChoice}
		\choice{\text{A}}
		\choice[correct]{\text{B}}
	\end{multipleChoice}
\end{question}

\begin{question}
Suppose you have \$2,000 in credit card debt. Your credit card charges an 18\% annual interest rate compounded monthly. (Round all answers to the nearest cent.)

What is the minimum amount you must pay in order to be sure that you will eventually pay off your debt? $\answer{30}$

How much do you end up paying in order to pay off this debt if you make a \$50 payment per month? $\answer{3076.84}$

How much do you end up paying in order to pay off this debt if you make a \$100 payment per month? $\answer{2394.24}$
\end{question}

\begin{question}
Suppose you want to buy a \$15,000 car. You make a down payment of \$3,000 and borrow the rest from a bank at a 4.2\% annual interest rate compounded monthly that must be paid off in 60 months. (Round all answers to the nearest cent.)
	
What is your monthly payment to ensure that you pay off your loan at the end of the 60 months? $\answer{222.08}$

What is the total interest for this loan? $\answer{1324.98}$

Recompute your answers if you are able to make a down payment of \$4,000.

New monthly payment. $\answer{203.58}$

New total interest. $\answer{1214.56}$
\end{question}

\begin{question}

\end{question}

\section{Workspace}

\begin{sageCell}
# Use this cell to solve the above questions.
\end{sageCell}

\end{document}
