\documentclass{ximera}  
\title{Matplotlib}  
\begin{document}  
\begin{abstract}  
We give an introduction to plotting using Matplotlib.
\end{abstract}  
\maketitle

\section{Matplotlib}

The \link[Matplotlib]{https://matplotlib.org/} library is a plotting library that is commonly used for producing publication quality figures of all kinds. Matplotlib has many features (too many to cover here). The easiest way to create a plot is to use the example gallery found on the Matplotlib website. Find the type of plot and features that you want and use the code sample provided.

To illustrate the kinds of plots that we will generate, consider the following example. The code below produces a plot of $f(x)=5x^2-400$ and $g(x)=x^3-x+1$ on the interval $[-10,10]$. 

\begin{verbatim}
==============================
import numpy as np
import matplotlib.pyplot as plt

x = np.linspace(-10,10,100)
y1 = 5*x**2-400
y2 = x**3-x+1
plt.plot(x,y1,color='blue',label='$f(x)$')
plt.plot(x,y2,color='red',label='$g(x)$')
plt.xlabel('$x$')
plt.ylabel('$y$')
plt.legend()
plt.title('A graph of $f(x)$ and $g(x)$')
plt.show()
==============================
\end{verbatim}

\begin{sageOutput}
import numpy as np
import matplotlib.pyplot as plt

x = np.linspace(-10,10,100)
y1 = 5*x**2-400
y2 = x**3-x+1
plt.plot(x,y1,color='blue',label='$f(x)$')
plt.plot(x,y2,color='red',label='$g(x)$')
plt.xlabel('$x$')
plt.ylabel('$y$')
plt.legend()
plt.title('A graph of $f(x)$ and $g(x)$')
plt.show()
\end{sageOutput}

The code above covers the main plotting features we will use in this course. The meaning of each line is covered below.

\begin{itemize}
	\item \verb|import matplotlib.pyplot as plt| - imports the appropriate Matplotlib functions with the commonly used alias \verb|plt|
	\item \verb|np.linspace(-10,10,100)| - produces an array containing 100 evenly spaced points from -10 to 10
	\item \verb|plt.plot| - adds the data points to the image, additional arguments specify their color and add a label for the plot legend (use \verb|plt.scatter| to draw points without connecting them)
	\item \verb|plt.xlabel|, \verb|plt.ylabel| - adds axes labels
	\item \verb|plt.legend| - adds a legend to the graph
	\item \verb|plt.title| - adds a title to the graph
	\item \verb|plt.show| - displays the graph
\end{itemize}

Note that the inputs for \verb|np.plot| can be regular lists. In this case we choose to use arrays due to ease of use.

\section{Problems}

Note that for the questions below, the hints contain the solutions.

\begin{question}
Use the Matplotlib library to plot $f(x)=e^{-x}\sin{x}$ and $g(x)=e^{-x}\cos(x)$ on $[0,2\pi]$.
	\begin{hint}
		\begin{sageCell}
import numpy as np
import matplotlib.pyplot as plt

x = np.linspace(0,2*np.pi,100)
y1 = np.exp(-x)*np.sin(x)
y2 = np.exp(-x)*np.cos(x)
plt.plot(x,y1,color='blue',label='$f(x)$')
plt.plot(x,y2,color='red',label='$g(x)$')
plt.xlabel('$x$')
plt.ylabel('$y$')
plt.legend()
plt.title('A graph of $f(x)$ and $g(x)$')
plt.show()
plt.clf() # This clears the figure to avoid drawing the same graph on the next plot.
\end{sageCell}
	\end{hint}
\end{question}

\begin{question}
	Use the Matplotlib library to create a scatter plot (using \verb|plt.scatter|) of $f(x) =x^2-x+1$ using 20 points on the interval $[-2,2]$.
	
	\begin{hint}
\begin{sageCell}
import numpy as np
import matplotlib.pyplot as plt

x = np.linspace(-2,2,20)
y = x**2-x+1
plt.scatter(x,y)
plt.xlabel('$x$')
plt.ylabel('$y$')
plt.title('A graph of $x^2-x+1$')
plt.show()
\end{sageCell}
	\end{hint}
\end{question}

\section{Workspace}

\begin{sageCell}
# Use this cell to solve the above questions.
\end{sageCell}

\end{document}
